\section{Lessons Learned}\label{sec:aslevel:lessons_learned}
In this chapter we characterized content delivery networks on autonomous systems level. %by evaluating measurements of the most common content delivery principles P2P and CDN.
For that purpose we summarized related work and used measurements conducted on the distributed platform PlanetLab and a crowdsourcing platform.
To assess the potential of peer assisted content delivery approaches, we determined the number of active IP-addresses from the Internet Census dataset.

%P2P -> CDN
%to accurately model content delivery networks ... is described.
%While ... we find three major outcomes.
First, we analyzed the Internet Census Dataset to derive the distribution of IP-addresses on autonomous systems.
To this end, we used a mapping of IP-addresses to autonomous system numbers.
we find that the distribution of IP-addresses is highly heterogeneous showing that 30\% of the active IPs belong to the 10 largest autonomous systems.
This means that the potential of approaches that use resources of home routers highly depend on the ISP network.

Second, we proposed the usage of crowdsourcing platforms for distributed network measurements to increase the coverage of vantage points.
We evaluated the capability to discover global networks by comparing the coverage of video server detected using a crowdsourcing platform as opposed to using the PlanetLab platform.
To this end, we used exemplary measurements of the global video CDN YouTube, conducted in both the PlanetLab platform as well as the crowdsourcing platform Microworkers.
Our results show that the vantage points of the concurring measurement platforms have very different characteristics.
We could show that the distribution of vantage points has high impact on the capability of measuring a global content distribution network.
The capability of PlanetLab to measure a global CDNs is rather low, since 80\% of requests are directed to the United States.
Our results confirm that the coverage of vantage points is increased by crowdsourcing.
Using the crowdsourcing platform we obtain a diverse set of vantage points that reveals more than twice as many autonomous systems deploying video servers than the widely used PlanetLab platform.
%Part of future work is to determine if the coverage of vantage points can be even further increased by targeting workers from specific locations to get representative measurement points for all parts of the world.

Finally, we have investigated where in the Internet BitTorrent traffic is located and which ISPs benefit from its optimization. To this end, we used measurements of live BitTorrent swarms to derive the location of BitTorrent peers and data provided by Caida.org in order to calculate the actual AS path between any two peers.
Our results show that the traffic optimization potential depends heavily on the type of ISP. Different ISPs will pursue different strategies to increase revenues.
Our results confirm that selecting peers based on their locality has a high potential to shorten AS paths between peers and to optimize the overlay network. In the observed BitTorrent swarms twice as much traffic can be kept intra-AS using locality peer selection. Thus, the inter-AS traffic is almost reduced by \unit[50]{\%} in \tier and in large ISPs.
%If ...
%This would for example allow ..

Based on the results obtained in this chapter, we develop models that describe the characteristics of CDNs and the number of active subscribers in ISP networks.
The models allow us to analyze the performance of traffic management mechanisms in realistic scenarios.
%Upper bounds / potential of p2p cdn approach.
%Including the application providers as stakeholders and considering their key performance indicators requires new models but also allows us to better understand the impact of mechanisms implemented in applications.
%Key performance indicators of application providers sometimes overlap with those relevant to users, as application providers try to improve the experience of users in order to reduce churn.

%Highlighted by both, the P2P approach and the CDN approach the distribution of peers on autonomous systems and the popularity of content is highly heterogeneous.
%Dynamics (orange paper)
%While general traffic management mechanisms intent to optimize the cdn
%this only increases the efficiency of the ISP cache, which may cause suboptimal results if the popularity of videos large variance.
%Thus, we suggest to  tradeoffs, within reason, to the user.
%This approach could be seen as extending the \emph{Economic Traffic Management} approach to the user.
