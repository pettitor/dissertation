\subsection{Distributed Active Measurement Description}
\label{sec:method}

To assess the capability of crowdsourcing for distributed active measurements we conduct measurements with both  PlanetLab and the commercial Crowdsourcing platform Microworkers~\cite{microworkers}.
We measure the global expansion of the YouTube CDN by resolving physical server IP-addresses for clients in different locations.

\subsubsection{Description of the PlanetLab Measurement}

PlanetLab is a publicly available test bed, which currently consists of 1173 nodes at 561 sites.
The sites are usually located at universities or research institutes.
Hence, they are connected to the Internet via NRENs.
To conduct a measurement in PlanetLab a slice has to be set up which consists of a set of virtual machines running on different nodes in the PlanetLab test bed.
Researchers can then access these slides to install measurement scripts.
In our case the measurement script implemented in Java extracted the server hostnames of the page of three predetermined YouTube videos and resolved the IP addresses of the physical video servers.
The IP addresses of the PlanetLab clients and the resolved IP addresses of the physical video servers were stored in a database.
To be able to investigate locality in the YouTube CDN, the geo-location of servers and clients is necessary.
For that purpose the IP addresses were mapped to geographic coordinates with MaxMinds GeoIP database \cite{geolite}.
The measurement was conducted on 220 randomly chosen PlanetLab nodes in March 2012.

\subsubsection{Description of the Crowdsourcing Measurement}
To measure the topology of the YouTube CDN from an end users point of view who is connected by an ISP network we used the crowdsourcing platform Microworker~\cite{microworkers}.
The workers were asked to access a web page with an embedded Java application, which automatically conducts client side measurements.
These include, among others, the extraction of the default and fallback server URLs from three predetermined YouTube video pages.
The extracted URLs were resolved to the physical IP address of the video servers locally on the clients.
The IP addresses of video servers and of the workers client were sent to a server which collected all measurements and stored them in a database.

In a first measurement run, in December 2011, 60 different users of Microworkers participated in the measurements.
Previous evaluation have shown, that the majority of the platform users is located in Asia~\cite{conf2011-410}, and accordingly most of the participants of there first campaign were from Bangladesh.
In order to obtain wide measurement coverage the number of Asian workers participating in a second measurement campaign, conducted in March 2012, was restricted.
In total, 247 workers from 32 different countries, finished the measurements successfully identifying 1592 unique physical YouTube server IP addresses.
