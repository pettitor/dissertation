\section{Lessons Learned}\label{sec:aggregation:lessonslearned}
In this chapter we investigate the potential of bandwidth aggregation approaches with offloading policy.
A direct application is the aggregation of backhaul link bandwidth to increase the overall capacity and cope with the increasing demand of traffic. %to reduce the load on cellular networks; carried by mobile networks, traffic is offloaded to Wi-Fi networks.

To this end, we develop a Markov model that consists of an M/M/n loss system for each link.
The offloading policy is modeled by introducing a support and an offloading threshold.
In parameter studies we investigate the impact of the threshold setting on the blocking probability and the received bandwidth.
The threshold settings implemented in the prototype BeWiFi with a support threshold of 70\% provides a good trade-off between sharing spare bandwidth and leaving capacity as buffer for peak periods.
Our results show that even if the cooperating system is overloaded, only up to 1\% of the bandwidth is lost in off peak periods.
%Cooperating systems benefit from aggregating bandwidth and can get close to the performance of complete sharing if thresholds are set accordingly.
The received bandwidth of a system can exceed its capacity significantly if the cooperating system is underutilized.

In practice the only limitation of bandwidth aggregation is the actual bandwidth available. The available bandwidth increases with the number of access links.
In order to evaluate the performance of a system with multiple access links that share their bandwidth, we approximate the steady state probabilities of a multi-dimensional Markov chain using a fixed point iterative procedure.

The full potential of the bandwidth aggregation approach is reached when an overloaded link can use the spare bandwidth of an underutilized link by exceeding its capacity significantly.
A joint fixed point iteration of an outer and an inner composite system is used to derive the state probabilities in heterogeneous load conditions.
In parameter studies we investigate the potential of the mechanism depending on the number of access links and find that in case of underutilized cooperating links, the bandwidth gain grows faster than linear with the number of contributing access links.
By prioritizing links, we show that the mechanism is robust against free riders, as cooperative users are not punished for sharing among only free riders, since only about 3\% of the bandwidth is lost in the considered case studies.
In addition, if each system would be egoistic no bandwidth can be shared and no one could profit.
Thus the system provides incentives to contribute to increase the overall system capacity.

Finally, we validate our model by simulation and consider deterministic and hyper-exponential service time distributions to assess the system performance in more general cases.
Our results show that only if the service time is highly variant and if one of the links is overloaded, only slightly more bandwidth is received by the overloaded link than in the model. Hence, the analytic model can also be used to assess the system performance with general service time distributions.
