\section{Simulation Model for Hierarchical Caching Systems}\label{sec:eval}

To evaluate the performance of a CDN supported by home routers, the paper simulates two scenarios. The first scenario simulates requests to a CDN with caches organized in a tree structure and compares isolated caches to cooperating caches to assess the benefit of the overlay.
The second scenario adds an AS topology with peering and customer-to-provider links to evaluate the inter-domain traffic saving potential. A customer-to-provider link exists between a customer ISP and its transit provider, if the customer ISP pays the transit provider to forward its traffic destined to parts of the Internet that the customer ISP does not own or cannot reach.

%\subsection{Caching}

To assess the benefit of the RB-HORST dependent on the number of shared home routers and the size of the ISP, the paper assumes and evaluates a tiered caching architecture with resource locations at three different tiers, including the main data center of the content provider, CDN caches, and end-user equipment.
%Table 6 3 shows the default parameters of the content delivery simulation.
The number of different content items to be downloaded or streamed from the resources is specified by the catalog size $N$. A Tier-1 resource is the data center of the content provider, where all $N$ content items are stored. Tier-2 resources are edge caches and ISP caches, typically organized in a CDN, which are located close to Internet exchange points or within ISP networks. Requests served by ISPs or edge caches produce less or no inter-domain costs. Thus, these caches are referred as ISP caches in the following. The capacity of ISP caches is given as a fraction of $N$ and is specified by $C_{ISP}$. The caching strategy of ISP caches is LRU.
%Each autonomous system hosts an ISP cache.
Within tier-3, the caches are placed on shared HRs that run the RB-HORST mechanism. These caches are referred to in the following as home routers (HRs). The cache capacity of HRs is specified by $C_{HR}$ and their caching strategy is LRU. In this study the $C_{HR}$ is set to four (4) content items.

The paper evaluates the performance dependent on the autonomous system size $n_{user}$, in terms of the number of end-users in the autonomous system. The probability that an end-user has RB-HORST installed and shares contents from its HR is given by $p_{share}$. The probability that a user requests certain content items depends on the content's popularity distribution, which is specified by the Zipf exponent $\alpha$.
