\section{Lessons Learned}\label{sec:cloud:lessons_learned}
To support content delivery networks, cache capacities on small data-centers with limited upload capacities, such as home routers, can be used to reduce the overall energy consumption and operation cost of the system.
%Such content delivery networks are crucial to meet the data challenge in mobile networks base stations
%get content even closer to users, utilize spare capacities on home routers and a the edge of the networks
In this chapter we evaluate the performance of hierarchical content delivery networks that have a high number of caches with limited capacity in the lowest tier.

To this end, we first provide a comprehensive overview of literature on performance evaluation of caching systems and show the state of the art of current research.
We describe the most relevant traffic models and caching strategies and introduce the Che approximation, which is a highly versatile method to determine the efficiency of caching systems.

We develop a simulation framework for hierarchical content delivery networks to evaluate the system characteristics that are not covered by the analytic models.
This allows us to evaluate the approach in terms of caching efficiency and inter-domain traffic.
The results show that an overlay is imperative for the success of such an approach, especially for a high number of small caches.
Moreover, by investigating the share of locally served content requests, the impact for the network operator is quantified.
The results indicate that such a mechanism significantly reduces the inter-domain traffic and the contribution of an operator-owned cache.
The ISP owned cache can be discontinued if at least every thousandth user shares its home router for caching in large ISPs.
The system was prototyped in \cite{rbhorst-demo} showing the practicability of the approach.

We develop an analytical model based on the Erlang formula for loss systems to evaluate the performance of hierarchical cache systems with small capacities and limited upload bandwidth.
The results show that the efficiency of the overlay can be even further increased by more than 10\% if an optimal content placement is used.
If the bandwidth is limited to 1.6Mbps the cache hit rate is reduced by about 30\%, which shows the importance of considering bandwidth constraints in the analysis.
There is a high potential to increase the efficiency of the content delivery network if only a small or no ISP cache is available.
If a larger ISP cache is available the benefit of the approach highly depends on the number of caches available and their upload bandwidth.
In the considered case at least 1000 tier-1 caches with more than 1.6Mbps upload bandwidth need to be available in a system with a large ISP cache so that the hit rate gain is more than 10\%.

%TODO REPHRASE: During off-peak operation of the network, the users can cheaply populate their caches with parts of popular contents.
