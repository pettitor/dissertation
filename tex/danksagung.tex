Prof. Dr. Tran-Gia hat die Rahmenbedingungen geschaffen, unter denen diese Dissertation entstehen konnte.
Unter seiner Leitung entstand am Lehrstuhl eine Atmosphäre, in der sowohl wissenschaftlicher Diskurs als auch ein freundliches Miteinander stattfinden konnte.
Er hat mir die Gelegenheit gegeben an wissenschaftlichen und industriellen Kooperationen im In- und Ausland teilzunehmen, die mich und den Inhalt dieser Dissertation maßgeblich geprägt haben.
Er selbst stand immer für fachliche Diskussionen zur Verfügung und war, insbesondere für die theoretischeren Aspekte dieser Arbeit, meine Anlaufstelle.
Ich danke ihm außerdem für das Vertrauen in mich, das mir diese Arbeit erst ermöglicht hat. %DONE

Ich danke natürlich auch meinem Zweitgutachter, Prof. Dr. Davoli, für die Bereitschaft meine Dissertation zu begutachten. %DONE

Den Mitgliedern der Prüfungskommission meiner Disputation, Prof. Dr. Kolla und Prof. Dr. Hotho, danke ich für die Flexibilität bei der Terminfindung und der freundlichen Atmosphäre während der Prüfung. %DONE

Dank gebührt natürlich auch Alison Wichmann für die Unterstützung bei der Bewältigung der zahlreichen administrativen Hürden im wissenschaftlichen und projektbezogenen Alltag. %DONE

Mit Prof. Dr. Wehnes konnte ich auch im Rahmen eines Industrieprojektes zusammenarbeiten. Aus seiner Erfahrung im Bereich Projektmanagement konnte ich persönlich viel Lernen.
Hierfür danke ich ihm. %DONE

Meine Kollegen Matthias Hirth,  Anh Nguyen-Ngoc, Nicholas Gray, Christopher Metter, Michael Seufert, Valentin Burger, Kathrin Borchert, Lam Dinh-Xuan, Dr. Florian Wamser, Steffen Gebert, Stanislav Lange und Dr. Thomas Zinner sind im Laufe der Arbeit am Lehrstuhl und insbesondere im Kaffeeraum zu Freunden geworden, für die ich in den letzten neun Monaten meiner Arbeitszeit die langen Zugfahrten zwischen Darmstadt und Würzburg gerne in Kauf genommen habe. %DONE

Auch außerhalb der Arbeitszeit am Lehrstuhl haben mich einige meiner Kollegen begleitet.
In vielen virtuellen Expeditionen haben mich Matthias Hirth und Valentin Burger durch Siege und Niederlage begleitet. %%% ein begleitet loswerden
Kathrin Borchert und Nicholas Gray waren immer gerne zu abendlichen Spielerunden bereit.
Mein Mitstreiter im Bereich DevOps, Steffen Gebert, hat mit mir Workshops zum Thema DevOps besucht, die ich zu den Highlights der von mir besuchten Konferenzen zähle.  
Euch allen vielen Dank für eure Freundschaft und die schönen Stunden auch außerhalb des Lehrstuhls. %DONE

Bedanken möchte ich mich auch bei meinen Gruppenleitern Dr. Rastin Pries, Prof. Dr. Tobias Hoßfeld und Dr. Florian Wamser.
Während meiner wissenschaftlichen Karriere standen sie mir stets mit Rat und Tat zur Seite, waren immer gerne bereit neue und interessante Fragestellungen zu diskutieren, haben mir aber auch, gerade gegen Ende meiner Zeit am Lehrstuhl, die notwendigen Freiheiten gegeben für mich eigene Themengebiete zu definieren. %DONE

Zahlreiche Studenten haben mich während meiner Zeit am Lehrstuhl unterstützt, sei es bei der Betreuung von Vorlesungen oder der Durchführung von Industrieprojekten.
Auch ihnen und den Studenten, mit denen ich im Rahmen von Seminar-, Praktikums- oder Abschlussarbeiten zusammenarbeiten konnte, gilt natürlich mein Dank. %DONE

Mit Dr. Florian Metzger konnte ich nur kurze Zeit am physikalisch selben Ort arbeiten.
Trotzdem war er immer gerne bereit Themen aus allen Bereichen der Informatik zu diskutieren und darüber hinaus auch privat stets ein guter Freund. %TODO: das geht anders 

Meine ehemalige Kollegen Dr. Michael Jarschel, Dr. Frank Lehrieder, Dr. Dominik Klein, Prof. Dr. Barbara Stähle, Prof. Dr. Dirk Stähle, Dr. Daniel Schlosser und Dr. Michael Duelli haben meinen wissenschaftlichen Arbeitsstil nicht unwesentlich, und natürlich nur positiv, beeinflusst. %TODO: mehr

Ich danke meinen Eltern, Sylvia und Bernd Schwartz, denn Sie haben mein Interesse an der Informatik schon während meiner Schulzeit erkannt und gefördert.
Sie haben mir mein Studium ermöglicht und damit wesentlich dazu beigetragen, dass ich meine Promotion erfolgreich durchführen konnte. %DONE

Nicht zuletzt danke ich meiner Frau, Prof. Dr. Alexandra Schwartz, und unserem Sohn Gabriel.
Insbesondere in den letzten Monaten der Erstellung dieser Arbeit bekamen sie mich, durch die langen Fahrzeiten nur viel zu selten zu Gesicht.
Trotzdem haben sie mich unterstützt und mir die Kraft gegeben, diese Arbeit zu einem guten Abschluss zu bringen.
Danke. %DONE